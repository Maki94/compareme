\section{Naivna Bajesova metoda - Naïve Bayes}
Probabalistička metoda za višeklasnu klasifikaciju. Temelji se Bajesovom pravilu uslovne verovatnoće. Naivna pretpostavkada jedna karakteristika objekata ne yavisi od drugis karakteristika. Bajesovo pravilo uslovne verovatnoće ima oblik P(A|B) = P(B|A)P(A)/P(B). P(A|B) je verovatnoća nastupanja događaja A pod uslovom da se desio događaj B. P(A) i P(B) su nezavisne verovatnoće nsatupanja događaja A, odnosno B. U zavisnosti od prirode podataka kojima se opisuju objekti i u zavisnosti od oblasti primene, postoje tri modela naivne Bajesove metode. Gausov model, Multinomijalni model, Bernulijev model

\subsection{Gausov model}

Koristi se za opštu klasifikaciju objekata koji su predstavljeni vektrorom realinih brojeva
Neka je objekat predstavljen vektorom slučajnih promenljivih (F1,F2,F3,..Fn)
Verovatnoća da objekat pripada klasi C na osnovu Bajesovog pravila je P(C|F1,F2,F3,...,Fn) = P(C)P(F1,F2,F3,...,Fn|C) /P(F1,F2,F3,...,Fn)

\subsection{Multinomijalni model}

Koristi se ya klasifikaciju objekata predstavljenih sekvencom prirodnih brojeva. \\
\subsection{Bernulijev model}

Koristi se ya klasifikaciju objekata predstavljenih binarnim vektorima. \\
